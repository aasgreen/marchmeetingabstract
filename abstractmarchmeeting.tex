
\documentclass{article} %
\usepackage{graphicx,amssymb} %
\usepackage[affil-it]{authblk}

\textwidth=15cm \hoffset=-1.2cm %
\textheight=25cm \voffset=-2cm %

\pagestyle{empty} %

\date{} %

\def\keywords#1{\begin{center}{\bf Keywords}\\{#1}\end{center}} %



\begin{document}

% Type down your paper title
\title{Novel Flow Meter System}

\author{Adam A. S. Green%
\thanks{\texttt{adam.green@colorado.edu}}}
\author{Cheol Park}
\author{Joe Macglennan}
\author{Matt Glaser}
\author{Noel Clark}

\affil{Department of Physics, University of Colorado, Boulder}
 
\date{Dated: \today}

\maketitle

\thispagestyle{empty}



% The abstract

\begin{abstract}
\noindent We report the realization of a novel flow meter.
 The gas-flow to be measured is channeled over a freely-suspended
film (FSF) of liquid crystal; by observing the velocity profile of the FSF the
flow rate of the gas can be extracted. The velocity profile of the FSF can be
accurately mapped by digitally tracking naturally occuring islands (these
islands are circular regions where extra liquid crystal bunches). Although our
flow meter is a simple application of the technology, it is also a demonstration
of a reliable test bed for 2D fluid physics.
\end{abstract}

\keywords{Fluid Dynamics, Flowmeter, Liquid Crystals} %




                    \end{document}

