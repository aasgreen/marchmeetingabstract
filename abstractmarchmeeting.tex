
\documentclass{article} %
\usepackage{graphicx,amssymb} %
\usepackage[affil-it]{authblk}
\usepackage{setspace}
\textwidth=15cm \hoffset=-1.2cm %
\textheight=25cm \voffset=-2cm %

\pagestyle{empty} %

\date{} %

\def\keywords#1{\begin{center}{\bf Keywords}\\{#1}\end{center}} %



\begin{document}

% Type down your paper title
\title{Novel Flow Meter System}

\author{Adam A. S. Green%
\thanks{Electronic address: \texttt{adam.green@colorado.edu}}}
\author{Cheol Park}
\author{Joe Maclennan}
\author{Matt Glaser}
\author{Noel Clark}

\affil{Department of Physics, University of Colorado, Boulder}
 
\date{Dated: \today}

\maketitle

\thispagestyle{empty}



% The abstract
\begin{doublespace}
\noindent We present the realization of a idealized 2D hydrodynamic system
coupled to a
gas jet, and show that this technology can be used as a novel flow meter.
Freely-Suspended Films (FSF) of liquid crystals are one of the closest physical realizations of
idealized 2D fluid flow. Because of the  existence of closed-form solutions for 2D
fluid flow, the velocity of a gas jet travelling over a channel of FSF can be
inferred by studying the velocity profile of the FSF. This velocity profile can be
directly imaged by digitally tracking the trajectory of islands of liquid
crystal material present in the FSF. The velocity profile is then fitted to the
closed-form solutions of 2D hydrodynamics, and the velocity of the gas jet can
then be extracted. Although the flow meter we present is a simple
application, it is also serves as a demonstration
of a robust test-bed for further exploration of 2D hydrodynamics.
\end{doublespace}
\vspace{20 mm}
\keywords{Fluid Dynamics, Flow meter, Liquid Crystals} %




                    \end{document}

